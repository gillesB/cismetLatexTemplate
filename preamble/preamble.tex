%
%	Preamble of Master Thesis 
%	TODO adaption to cismet layout
%
%	KOMA-Script Dokumentklasse "scrbook"

% prints warnings about obsolete syntax and syntax which will probably break
\RequirePackage[l2tabu, orthodox]{nag}

%\documentclass[12pt, pdftex, a4paper, bibliography=totoc, listof=totoc, parskip=half, abstract=on]{scrreprt}
\documentclass[12pt, twoside, openright, pdftex, a4paper, bibliography=totoc, listof=totoc, parskip=half, abstract=on]{scrreprt}
\raggedbottom

% use T1 Fonts
% The Latin Modern fonts are enhanced versions of the Computer Modern fonts. They have enhanced metrics and glyph coverage.
\usepackage{lmodern}
\usepackage[T1]{fontenc}

% use utf8 as encoding
\usepackage[utf8]{inputenc}

% Language configuration
% quotes
\usepackage{csquotes}

% comment out, which is not needed
%\usepackage[english]{babel}
\usepackage[ngerman]{babel}

% mathematical symbols
\usepackage{amssymb}

%%%% === Verzeichnisse (TOC, LOF, LOT, BIB) ===
   %liststotoc,      % Tabellen & Abbildungsverzeichnis ins TOC
   %idxtotoc,        % Index ins TOC
   %bibtotoc,         % Bibliographie ins TOC
   %bibtotocnumbered, % Bibliographie im TOC nummeriert
   %liststotocnumbered, % Alle Verzeichnisse im TOC nummeriert 

\usepackage{color} 
\definecolor{lightgray}{rgb}{.9,.9,.9}
\definecolor{darkgray}{rgb}{.4,.4,.4}
\definecolor{purple}{rgb}{0.65, 0.12, 0.82}


\usepackage{tabulary}
\usepackage{colortbl}



% format for tables and image descriptions
\addtokomafont{caption}{\small\bfseries}
\addtokomafont{captionlabel}{\bfseries}
 
% line spacing 1,5
\usepackage{setspace}
\onehalfspacing

% page settings
\usepackage{geometry}
\geometry{a4paper,left=35mm,right=25mm,%
bottom=25mm,top=25mm,bindingoffset=5mm,%
includehead,includefoot
}
% allow landscape pages
\usepackage{pdflscape}

%Fußnoten durchlaufend nummerieren
\usepackage{chngcntr}
\counterwithout{footnote}{chapter}

\pagestyle{headings}
% Eigene Kopfzeile
%\usepackage{scrpage2}
%\pagestyle{scrheadings}
%\automark[section]{chapter}


% url – Ver­ba­tim with URL-sen­si­tive line breaks
\usepackage{url}

%PDF Version 1.6
\pdfminorversion=6

% for inserting graphics
\usepackage{graphicx}
\usepackage{caption}
\usepackage{subcaption}

% for inserting graphics as background picture
\usepackage{eso-pic}

\usepackage{tocbasic}

% configuration bibliography

% As backend use biber
% Biber: A BibTeX replacement for users of BibLaTeX
% Deals with the full range of UTF-8 
% http://biblatex-biber.sourceforge.net/
	
%\usepackage[backend=biber,style=numeric]{biblatex} %[5]
\usepackage[backend=biber,style=alphabetic]{biblatex} %[JW86]

% more styles can be found in the deocumentation of biblatex

\bibliography{bibArchive/libarchive}


\usepackage{color}
\usepackage{xcolor}
\usepackage{listings}
\usepackage{caption}

\DeclareCaptionFont{white}{\color{white}}
\DeclareCaptionFormat{listing}{\colorbox{gray}{\parbox{\textwidth}{#1#2#3}}}
\captionsetup[lstlisting]{format=listing,labelfont=white,textfont=white}

%----------------------------------
%----------JavaScript--------------
%----------------------------------
\lstdefinelanguage{JavaScript}{
  keywords={typeof, new, true, false, catch, function, return, null, catch, switch, var, if, in, while, do, else, case, break},
  keywordstyle=\color{blue}\bfseries,
  ndkeywords={class, export, boolean, throw, implements, import, this},
  ndkeywordstyle=\color{darkgray}\bfseries,
  identifierstyle=\color{black},
  sensitive=false,
  comment=[l]{//},
  morecomment=[s]{/*}{*/},
  commentstyle=\color{purple}\ttfamily,
  stringstyle=\color{blue}\ttfamily,
  morestring=[b]',
  morestring=[b]"
}

\lstset{
	    language=JavaScript,
         basicstyle=\footnotesize\ttfamily, % Standardschrift
         numbers=left,               % Ort der Zeilennummern
         numberstyle=\tiny,          % Stil der Zeilennummern
         %stepnumber=2,               % Abstand zwischen den Zeilennummern
         numbersep=5pt,              % Abstand der Nummern zum Text
         tabsize=2,                  % Groesse von Tabs
         extendedchars=true,         %
         breaklines=true,            % Zeilen werden Umgebrochen
         keywordstyle=\color{red},
    		frame=b,         
 %        keywordstyle=[1]\textbf,    % Stil der Keywords
 %        keywordstyle=[2]\textbf,    %
 %        keywordstyle=[3]\textbf,    %
 %        keywordstyle=[4]\textbf,   \sqrt{\sqrt{}} %
         %stringstyle=\color{blue}\ttfamily, % Farbe der String
         showspaces=false,           % Leerzeichen anzeigen ?
         showtabs=false,             % Tabs anzeigen ?
         xleftmargin=17pt,
         framexleftmargin=17pt,
         framexrightmargin=5pt,
         framexbottommargin=4pt,
         %backgroundcolor=\color{lightgray},
         showstringspaces=false      % Leerzeichen in Strings anzeigen ?        
 }

\def\inline{\lstinline[basicstyle=\ttfamily,breaklines=true,breakatwhitespace,keywordstyle={}, 
literate=  {\\\-}{}{0\discretionary{}{}{}}]}


%-----------------------------------------------

%----------------------------------
%----------Neue Kommandos----------
%----------------------------------

\newcommand{\myfig}[1]{Abbildung~\vref{#1}}
\newcommand{\mytab}[1]{Tabelle~\vref{#1}}
\newcommand\todo[1]{\textcolor{red}{TODO: #1}}
\hyphenation{TODO}

%add the end of the line '\tabularnewline' has to be used insted of '\\'
\newcommand{\head}[1]{\centering \textcolor{white}{\textbf{#1} }}
\newcommand{\tableSection}[1]{\multicolumn{3}{|l|}{\rowcolor[gray]{0.9}\textbf{#1}}}

%Percent Value
\newcommand{\per}[1]{{#1}\,\%}


%-----------Abkürzungsverzeichnis---------------

%there are two ways to make abbreviations
%the traditional way
%\usepackage[intoc]{nomencl}
% Befehl umbenennen in abbr
%\let\abbr\nomenclature
% Deutsche Überschrift
%\renewcommand{\nomname}{Abkürzungsverzeichnis}
% Punkte zw. Abkürzung und Erklärung
%\setlength{\nomlabelwidth}{.20\hsize}
%\renewcommand{\nomlabel}[1]{#1 \dotfill}
% Zeilenabstände verkleinern
%\setlength{\nomitemsep}{-\parsep}
%\makenomenclature

%Gilles prefered way
%normal abbrev can be generated with \ac. \ac{abbr.} will result in Abbreviation (abbr.)
\usepackage[printonlyused]{acronym}


% hyperref should be one of the last imported packages
% although there are exceptions: http://tex.stackexchange.com/questions/1863/which-packages-should-be-loaded-after-hyperref-instead-of-before
% varioref – Intelligent page references
\usepackage{varioref}

\usepackage{hyperref}

% remove ugly red boxes around the hyperrefs
\hypersetup{%
    pdfborder = {0 0 0}
}


%random stuff

% prints the Euro sign
% \euro generates the symbol. 
% \EUR{x} shows an amount with the symbol and has the correct unbreakable thin space in between.
% http://www.theiling.de/eurosym.html
\usepackage{eurosym}

%----------------------------------
%----------------------------------
%----------------------------------


%
%	EOF
%
